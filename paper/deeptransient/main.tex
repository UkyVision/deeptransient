\documentclass{article}
\usepackage{spconf,amsmath,graphicx}

% Example definitions.
% --------------------
\def\x{{\mathbf x}}
\def\L{{\cal L}}

% Title.
% ------
\title{Deep Methods for Estimating Transient Scene Properties}
%
% Single address.
% ---------------
\name{}
\address{University of Kentucky}
%
% For example:
% ------------
%\address{School\\
%	Department\\
%	Address}
%
% Two addresses (uncomment and modify for two-address case).
% ----------------------------------------------------------
%\twoauthors
%  {A. Author-one, B. Author-two\sthanks{Thanks to XYZ agency for funding.}}
%	{School A-B\\
%	Department A-B\\
%	Address A-B}
%  {C. Author-three, D. Author-four\sthanks{The fourth author performed the work
%	while at ...}}
%	{School C-D\\
%	Department C-D\\
%	Address C-D}
%
\begin{document}
%\ninept
%
\maketitle
%
\begin{abstract}

  Deep learning has been used to obtain state of the art results in
  problems ranging from object classification, object detection, and
  scene classification. We evaluate the use of deep learning to
  predict more subtle visual scene aspects, such as the weather, the
  season, and subjective scene properties. 

\end{abstract}
%
\begin{keywords}
\end{keywords}

\section{Introduction}

% general intro to the problem and why it is important

% what do we propose to do? 

% how do we evaluate this? 
We evaluate the proposed method on several benchmarks 

% what we can do now that we have this?

% key contributions: repeat what was said above in bullet point form
The key contributions of this work are:
\begin{itemize}

  \item comparison of several deep learning network architectures for
    the problem

  \item state-of-the-art results on two benchmark datasets (transient
    attributes and two class weather)

  \item something with running it on webcam data?

\end{itemize}

\subsection{Related Work}

Previous approaches to this problem include (some general description
of previous approaches and how they compare to what we propose).

List of related papers:
\begin{itemize}

  \item add citation and one sentence description of what it does, one
    sentence description of how it is different from what we do, maybe
    something about how they evaluate.

\end{itemize}

\section{Estimating Transient Attributes Using CNNs}

\subsection{Recap of CNNs}

TODO: add a summary of the different models, and how they were
trained, include a bibtex citation for each one.

\subsection{Description of how we train ours}

TODO: write me

\subsection{Deep Network Visualization}

TODO: pictures of what the network looks like\dots highlighting how it
is different from those trained for other purposes... probably
focusing on the one or two that work the best

\section{Evaluation}

%Suggested plots:
%\begin{itemize}
%
%  \item summary results you have already shown
%
%  \item example good results, and plausible bad results
%
%\end{itemize}

\section{Conclusions}

TODO: write me

more accurate

simpler and easier to train

faster

composable


\begin{figure}[t]
	\centering
		\includegraphics[width=0.5\textwidth]{figs/caffenet_avg_err.png}
		\caption{Sorted errors from caffenet}
\end{figure}

\begin{figure}[t]
	\centering
		\includegraphics[width=0.5\textwidth]{figs/rel_err.png}
		\caption{Relative errors us minus them}
\end{figure}

\begin{figure*}[t]
	\centering
		\includegraphics[width=1.0\textwidth]{figs/avg_err_compare.png}
		\caption{comparing average errors}
\end{figure*}

\begin{figure*}[t]
	\centering
		\includegraphics[width=1.0\textwidth]{figs/fig_1.png}
		\caption{These are some evaluation curves. needs to be full top page}
\end{figure*}

%\bibliographystyle{IEEEbib}
%\bibliography{refs}

\end{document}

